\documentclass[11pt]{article}

% -----------------------------------------------------
% Packages
% -----------------------------------------------------
\usepackage{amsmath,amssymb,amsthm,amsfonts}
\usepackage{bm}
\usepackage{stmaryrd}
\usepackage{hyperref}
\usepackage{geometry}
\usepackage{enumitem}
\usepackage{tikz}
\usepackage{tikz-cd}
\usepackage{tikz-uml}
\usepackage{mathrsfs}
\usepackage{bbm}
\usepackage{mathtools}
\usepackage{listings}

\geometry{margin=1in}
\setlength{\parskip}{0.8em}
\setlength{\parindent}{0pt}

\title{\Large\bfseries
POLYXAN--RSVP STARSPACE \\
A Formal Specification for a Xanadu--RSVP Social Hyperstructure \\
with Multi-Galaxy Semantic Dynamics and Generative Field Algorithms
}
\author{Flyxion \and System Specification Draft}
\date{2025}

\begin{document}

\maketitle

\begin{abstract}
This document specifies the Polyxan--RSVP Starspace System: a hybrid
Xanadu-style hypermedia architecture, an RSVP generative substrate
(Scalar--Vector--Entropy fields) driving semantic and dynamical evolution,
and a starspace MMO interface in which each user occupies a partially
isolated galaxy region.
We define content atoms, spans, typed bidirectional links, media quines
generated by a Polycompiler, semantic force embeddings, user-galaxy
sheaves, RSVP field evolution, and the global \texttt{g}-reset operator.
We give a Lagrangian formulation of RSVP fields on semantic space and
their coupling to the hypergraph, a category-theoretic architecture,
an operational semantics for resets, a database schema and API sketch,
simulation pseudocode, UML diagrams, and verification-oriented invariants
suitable for mechanized proof assistants.
\end{abstract}

\tableofcontents

%%%%%%%%%%%%%%%%%%%%%%%%%%%%%%%%%%%%%%%%%%%%%%%%%%%%%%%%%%%%%%%%%%%%%%
\section{System Overview}

The Polyxan--RSVP Starspace system integrates:

\begin{itemize}[leftmargin=2em]
  \item A \textbf{Xanadu-style hypergraph} of Content Atoms with fine-grain spans.
  \item The \textbf{RSVP generative field system}: scalar $\Phi$, vector $\mathbf{v}$,
    entropy $S$, producing semantic gradients, cluster morphologies,
    and viewpoint curvature over a latent semantic manifold.
  \item A \textbf{semantic latent space} embedded in $\mathbb{R}^3$ as a star map,
    with N-body relaxation guided by RSVP fields.
  \item A \textbf{galaxy-shard universe}: each user $u$ sees a localized galaxy
        generated as a sheaf section over the global semantic space $X$.
  \item A \textbf{ship-projection MMO layer}: users appear as anonymous triangular
        ships; projections are holographic and non-destructive.
  \item A \textbf{global reset operator} triggered by holding key \texttt{g} for five seconds,
        recomputing the embedding and galaxy layouts under RSVP constraints.
  \item \textbf{Autoblink} stability constraints to keep certain users' local patches
        approximately invariant through resets.
\end{itemize}

%%%%%%%%%%%%%%%%%%%%%%%%%%%%%%%%%%%%%%%%%%%%%%%%%%%%%%%%%%%%%%%%%%%%%%
\section{Core Data Types}

\subsection{Content Atoms}

A Content Atom is the basic unit of meaning, regardless of media type.
Formally:

\[
\text{Atom} :=
  \{\text{id}: \mathbb{N},\
    \text{media}: M,\
    \text{payload}: B,\
    \text{tags}: \mathcal{P}(T),\
    \text{version} \in \mathbb{N},\
    \text{polyGroup} \in \mathbb{N} \cup \{\varnothing\}
  \}.
\]

\noindent
Here $M$ is the set of media types (text, audio, video, image, code, composite),
$B$ is a blob reference (to object storage or stream), and $T$ is a tag alphabet
(topics, languages, etc.).

%%%%%%%%%%%%%%%%%%%%%%%%%%%%%%%%%%%%%%%%%%%%%%%%%%%%%%%%%%%%%%%%%%%%%%
\subsection{Spans}

A Span provides fine-grained addressability.

\[
\text{Span} = (\text{spanId}, \text{atomId}, s, e)
\]

with $s<e$ and $s,e$ representing byte or time offsets within the payload.

%%%%%%%%%%%%%%%%%%%%%%%%%%%%%%%%%%%%%%%%%%%%%%%%%%%%%%%%%%%%%%%%%%%%%%
\subsection{Typed Links}

A Typed Link is an edge:

\[
L = (\text{linkId}, \text{from}, \text{to}, \tau, \text{creator}, t)
\]

where:
\begin{itemize}[leftmargin=2em]
  \item $\text{from},\text{to}$ are Spans or Atoms.
  \item $\tau \in \Lambda$ is a link type (reply, critique, support,
        transclusion, translation, summary, remix, etc.).
  \item $\text{creator}$ is the persona that authored the link.
  \item $t$ is a timestamp.
\end{itemize}

We maintain bidirectionality by ensuring that for each $L: X\to Y$,
adjacency structures store both outgoing and incoming references.

%%%%%%%%%%%%%%%%%%%%%%%%%%%%%%%%%%%%%%%%%%%%%%%%%%%%%%%%%%%%%%%%%%%%%%
\section{Polycompiler and Media Quines}

The Polycompiler is a system service:

\[
\text{Polycompile}: \text{Atom} \times \Sigma \to \text{Atom}
\]

where $\Sigma$ is a specification of the target modality or transformation
(e.g. (\emph{summary}, 1-minute video), (\emph{translation}, es\_MX), etc.).

All media variants $\{A_\sigma\}$ of a seed Atom $A$ share a common
polyGroup identifier $g$.

\begin{definition}[Media Quine]
A \emph{media quine} is a polyGroup $G$ of Atoms such that
for every $A_i \in G$ there exists $\sigma$ with
$\mathrm{Polycompile}(A_i,\sigma) \approx A_j$ for some $A_j \in G$,
up to a specified semantic equivalence relation $\approx$.
\end{definition}

%%%%%%%%%%%%%%%%%%%%%%%%%%%%%%%%%%%%%%%%%%%%%%%%%%%%%%%%%%%%%%%%%%%%%%
\section{Semantic Latent Space and Star Map}

Each semantic entity $x$ (Atom, PolyGroup, Persona, Topic cluster) has
an embedding:

\[
\mathbf{z}_x \in \mathbb{R}^d.
\]

This embedding is constructed from:
\begin{itemize}
  \item multimodal content encoders ($f_{\text{mm}}$),
  \item graph-structural features (typed links, centrality),
  \item RSVP field values $(\Phi, \mathbf{v}, S)$ at the node.
\end{itemize}

We then define a projection:

\[
\pi : \mathbb{R}^d \to \mathbb{R}^3
\]

which is a parametric, time-stable projection onto coordinates interpreted
as star positions:

\[
\mathbf{x}_x := \pi(\mathbf{z}_x).
\]

An N-body relaxation step adjusts $\mathbf{x}_x$ to satisfy aesthetic
and semantic constraints (e.g. cluster compactness, repulsion between
blocked regions), while preserving the relative ordering implied by
$\mathbf{z}_x$ and the RSVP fields.

%%%%%%%%%%%%%%%%%%%%%%%%%%%%%%%%%%%%%%%%%%%%%%%%%%%%%%%%%%%%%%%%%%%%%%
\section{Galaxy-Shard Architecture}

Let $X$ denote the global semantic manifold (the embedding space).
For each user $u$, define a center $\mathbf{z}_u$ corresponding
to the persona embedding, and an open neighborhood:

\[
U_u := B_R(\mathbf{z}_u) \subset X
\]

for some radius $R$ in the latent metric.
The \emph{galaxy map} for user $u$ is:

\[
\mathcal{G}_u := \{ (\mathbf{x}_x, x) \mid x \in X,\ \mathbf{z}_x \in U_u \}.
\]

Semantic isolation arises because traveling from $U_u$ to another user $v$'s
neighborhood $U_v$ requires multiple in-game steps: ship movement at finite
speed across $\mathbf{x}$-space.

%%%%%%%%%%%%%%%%%%%%%%%%%%%%%%%%%%%%%%%%%%%%%%%%%%%%%%%%%%%%%%%%%%%%%%
\section{User Ships, Projection, and Anonymity}

Each persona $p$ is displayed in its own galaxy as a triangular ship
at position:

\[
\mathbf{x}_p = \pi(\mathbf{z}_p).
\]

When $p$ projects into another user's galaxy, a \emph{ghost ship} representation
appears (triangle with no explicit username), subject to privacy rules.

\begin{definition}[Holographic Projection]
A holographic projection from user $u$ to galaxy $v$ is a morphism:
\[
\mathsf{Proj}_{u\to v}: \mathcal{G}_v \to \mathcal{G}_v
\]
that augments $\mathcal{G}_v$ with an anonymous ship entity $s_u$,
allowing $u$ to \emph{observe} $\mathcal{G}_v$ but not mutate its contents.
\end{definition}

All content created during projection is anchored in $u$'s own galaxy
(contexts, groups), but may link to spans originating from $\mathcal{G}_v$.

%%%%%%%%%%%%%%%%%%%%%%%%%%%%%%%%%%%%%%%%%%%%%%%%%%%%%%%%%%%%%%%%%%%%%%
\section{The \texttt{g}-Key Reset Operator and Autoblink}

Holding key \texttt{g} for 5 seconds triggers a \emph{Reset Event}:

\[
\mathsf{Reset}: \Sigma \to \Sigma'
\]

where $\Sigma$ is the full system state:
embeddings $\mathbf{z}_x$, field values $(\Phi,\mathbf{v},S)$, layout
positions $\mathbf{x}_x$, and galaxy views $\mathcal{G}_u$.

\subsection{Global Reset Transformation}

We can formalize reset as:

\[
\mathbf{z}_x' = \mathcal{R}_z(\mathbf{z}_x,\Phi,\mathbf{v},S),
\quad
\mathbf{x}_x' = \mathcal{R}_x(\mathbf{z}_x'),
\]

with the following constraints:

\begin{itemize}
  \item $\mathcal{R}_z$ respects RSVP dynamics (Section \ref{sec:rsvp-lagrangian}),
    so embeddings adjust according to updated fields.
  \item $\mathcal{R}_x$ is a new N-body relaxation seeded by $\mathbf{z}_x'$.
\end{itemize}

\subsection{Autoblink Constraint}

Users with \texttt{autoblink} enabled impose a local stability constraint:

\[
\|\mathbf{x}_u' - \mathbf{x}_u\| \le \epsilon
\]

for some small $\epsilon>0$. In the relaxation solver, these points become
soft constraints or pinned nodes; other stars flow around them.

%%%%%%%%%%%%%%%%%%%%%%%%%%%%%%%%%%%%%%%%%%%%%%%%%%%%%%%%%%%%%%%%%%%%%%
\section{RSVP--Polyxan Lagrangian}
\label{sec:rsvp-lagrangian}

We now define a Lagrangian for the RSVP fields over the semantic manifold $X$
and couple it to the Polyxan content graph.

\subsection{Fields and Densities}

Let $X$ be a Riemannian manifold representing semantic space with metric $g$.
Over $X$ we define:

\[
\Phi: X \times \mathbb{R} \to \mathbb{R},\quad
\mathbf{v}: X \times \mathbb{R} \to T X,\quad
S: X \times \mathbb{R} \to \mathbb{R}.
\]

Define a node density $\rho: X \to \mathbb{R}_{\ge 0}$ induced by the content graph,
e.g. via kernel smoothing over embeddings.
Define a link curvature scalar $\kappa: X \to \mathbb{R}$ that measures
non-local connectivity complexity (e.g. triangle density, motif structure).

\subsection{Action Functional}

We propose an action:

\[
\mathcal{A}[\Phi,\mathbf{v},S]
=
\int_{\mathbb{R}} dt \int_X d\mu_g \ \mathcal{L}(\Phi,\partial_t\Phi,\nabla\Phi,\mathbf{v},\nabla\mathbf{v},S,\nabla S;\rho,\kappa)
\]

with Lagrangian density:

\begin{align*}
\mathcal{L}
&=
\underbrace{
\frac{1}{2}(\partial_t \Phi)^2
 - \frac{c_\Phi^2}{2} \|\nabla \Phi\|^2
}_{\text{scalar kinetic/elastic}}\\
&\quad +
\underbrace{
\frac{1}{2}\|\partial_t \mathbf{v}\|^2
 - \frac{c_v^2}{2}\|\nabla \mathbf{v}\|^2
}_{\text{vector field kinetic/elastic}}\\
&\quad +
\underbrace{
\frac{1}{2}(\partial_t S)^2
 - \frac{c_S^2}{2}\|\nabla S\|^2
}_{\text{entropy field smoothing}}\\
&\quad - V(\Phi,\mathbf{v},S;\rho,\kappa),
\end{align*}

where $V$ is a potential encoding:

\begin{itemize}
  \item attraction of $\Phi$ to high-density regions (cluster formation),
  \item negentropic flows (alignment of $\mathbf{v}$ with $\nabla\Phi$),
  \item entropy minimization in well-structured semantic neighborhoods,
  \item penalties for excessive curvature $\kappa$ (graph over-complexity).
\end{itemize}

Variation of $\mathcal{A}$ yields Euler--Lagrange equations for the fields,
which can be discretized on the embedding graph.

\subsection{Coupling to Graph Nodes}

Each Atom $x$ sits at an embedding $\mathbf{z}_x \in X$.
We define the field values at node $x$ by restriction:
$\Phi_x(t) = \Phi(\mathbf{z}_x,t)$, etc.
A simple discrete evolution for embeddings is:

\[
\frac{d\mathbf{z}_x}{dt}
=
-\alpha \nabla \Phi(\mathbf{z}_x,t)
+\beta\,\mathbf{v}(\mathbf{z}_x,t)
-\gamma \nabla S(\mathbf{z}_x,t),
\]

where $\alpha,\beta,\gamma$ are hyperparameters governing attraction to semantic wells,
vector-flow drift, and entropy smoothing.

%%%%%%%%%%%%%%%%%%%%%%%%%%%%%%%%%%%%%%%%%%%%%%%%%%%%%%%%%%%%%%%%%%%%%%
\section{Category-Theoretic Architecture}

We now sketch a categorical view.

\subsection{Content Category}

Define a category $\mathbf{C}$:

\begin{itemize}
  \item Objects: Content Atoms and Spans.
  \item Morphisms: Typed Links $L: X\to Y$.
\end{itemize}

Composition is given by path concatenation when link types are composable;
identity morphisms are trivial self-links.

\subsection{Polycompiler as Endofunctor}

The Polycompiler induces an endofunctor:

\[
\mathsf{Poly}: \mathbf{C} \to \mathbf{C}
\]

that:

\begin{itemize}
  \item on objects: sends an Atom $A$ to a PolyGroup object, or to a specific
        media variant $A_\sigma$.
  \item on morphisms: lifts links along media quine equivalences, preserving
        semantic type when possible.
\end{itemize}

\subsection{RSVP Functor}

Define a functor:

\[
\mathsf{RSVP}: \mathbf{C} \to \mathbf{F}
\]

where $\mathbf{F}$ is a category of field configurations, e.g.:

\begin{itemize}
  \item objects: triples $(\Phi,\mathbf{v},S)$ defined on finite subsets of $X$,
  \item morphisms: restriction maps and field reparameterizations.
\end{itemize}

$\mathsf{RSVP}$ maps content/link structure into field source terms (e.g. node
densities, curvature contributions), and in turn field evolution feeds back into
embedding updates.

\subsection{Galaxy Sheaf}

Over $X$ we define a presheaf $\mathscr{G}$:

\[
\mathscr{G}(U) = \{\text{all galaxy renderings over }U\}
\]

with restriction maps $\rho_{UV}: \mathscr{G}(U)\to\mathscr{G}(V)$
for $V\subset U$.

\begin{theorem}[Sheaf Condition (Sketch)]
If:
\begin{itemize}
  \item the projection $\pi$ is deterministic and smooth,
  \item RSVP fields are continuous on $X$,
  \item content IDs and links are globally unique,
\end{itemize}
then $\mathscr{G}$ is a sheaf: compatible local views glue uniquely
to a global galaxy rendering.
\end{theorem}

\begin{proof}[Idea]
Galaxy renderings are determined by $(\mathbf{z}_x,\pi)$ and field values.
Compatibility on overlaps corresponds to agreeement on shared nodes and
their local layout under the same $\pi$ and field configuration.
Uniqueness follows from determinism of the layout algorithm.
\end{proof}

%%%%%%%%%%%%%%%%%%%%%%%%%%%%%%%%%%%%%%%%%%%%%%%%%%%%%%%%%%%%%%%%%%%%%%
\section{Operational Semantics of \texttt{g}-Reset}

We describe small-step semantics for the \texttt{g} key in a simplified form.

\subsection{State}

Let a system state be:

\[
\Sigma = (\mathcal{C},\mathbf{Z},\mathbf{X},F,\mathcal{U})
\]

where:
\begin{itemize}
  \item $\mathcal{C}$ = content graph (Atoms, Spans, Links),
  \item $\mathbf{Z} = \{\mathbf{z}_x\}$ embeddings,
  \item $\mathbf{X} = \{\mathbf{x}_x\}$ layout positions,
  \item $F = (\Phi,\mathbf{v},S)$ RSVP fields,
  \item $\mathcal{U}$ = user metadata (autoblink flags, ship positions).
\end{itemize}

\subsection{Events}

We introduce an event $\mathsf{GPress}(u,t)$ for a user $u$ holding \texttt{g}
from time $t$ to $t+\Delta$.

We define two rules:

\medskip
\noindent
\textbf{Broadcast Rule (short press).}
If $0<\Delta < 5\mathrm{s}$:

\[
\frac{}{\Sigma \xrightarrow{\mathsf{GPress}(u,\Delta)} \Sigma'}
\]

where $\Sigma'$ has $\mathcal{U}'$ updated to broadcast $u$'s current ship
position in nearby galaxies for some time window, but no change to $\mathbf{Z},\mathbf{X},F$.

\medskip
\noindent
\textbf{Reset Rule (long press).}
If $\Delta \ge 5\mathrm{s}$:

\[
\frac{}{\Sigma \xrightarrow{\mathsf{GPress}(u,\Delta)} \Sigma''}
\]

where:

\begin{align*}
\mathbf{Z}'' &= \mathcal{R}_z(\mathbf{Z},F,\mathcal{C}) \\
F'' &= \mathcal{R}_F(F,\mathcal{C}) \\
\mathbf{X}'' &= \mathcal{R}_x(\mathbf{Z}'',F'',\mathcal{U}) \\
\mathcal{U}'' &= \mathcal{U} \text{ (up to transient fields)}
\end{align*}

and $\mathcal{R}_x$ respects autoblink constraints by pinning or softly constraining
selected user positions.

%%%%%%%%%%%%%%%%%%%%%%%%%%%%%%%%%%%%%%%%%%%%%%%%%%%%%%%%%%%%%%%%%%%%%%
\section{Database Schema and API Sketch}

\subsection{Relational/Core Schema (Sketch)}

Tables (conceptual):

\begin{itemize}
  \item \texttt{atoms}(\texttt{id}, \texttt{media\_type}, \texttt{payload\_ref},
        \texttt{version}, \texttt{poly\_group\_id}, \texttt{created\_at}, \texttt{author\_id})
  \item \texttt{spans}(\texttt{id}, \texttt{atom\_id}, \texttt{start}, \texttt{end})
  \item \texttt{links}(\texttt{id}, \texttt{from\_span\_id}, \texttt{to\_span\_id},
        \texttt{link\_type}, \texttt{creator\_id}, \texttt{created\_at})
  \item \texttt{poly\_groups}(\texttt{id}, \texttt{root\_atom\_id})
  \item \texttt{personas}(\texttt{id}, \texttt{user\_id}, \texttt{name}, \texttt{avatar\_atom\_id})
  \item \texttt{embeddings}(\texttt{entity\_id}, \texttt{entity\_type}, \texttt{vector})
  \item \texttt{galaxy\_views}(\texttt{user\_id}, \texttt{center\_embedding},
        \texttt{params}, \texttt{last\_updated})
  \item \texttt{rsvp\_fields}(\texttt{patch\_id}, \texttt{phi\_params},
        \texttt{v\_params}, \texttt{s\_params})
  \item \texttt{reset\_events}(\texttt{id}, \texttt{trigger\_user\_id}, \texttt{at\_time})
\end{itemize}

Embeddings can be stored in a vector-capable store or separate service.

\subsection{API Sketch}

Representative endpoints (REST or gRPC-ish):

\begin{itemize}
  \item \verb|GET /atoms/{id}| -- fetch Atom metadata and (optionally) payload.
  \item \verb|POST /atoms| -- create Atom.
  \item \verb|POST /links| -- create Typed Link.
  \item \verb|POST /polycompile| -- request Polycompiler to generate variants.
  \item \verb|GET /galaxy/{userId}| -- fetch GalaxyView for user.
  \item \verb|POST /galaxy/{userId}/project| -- project to another user's galaxy.
  \item \verb|POST /events/gpress| -- notify backend of \texttt{g} press; backend
        decides whether to broadcast or reset.
  \item \verb|GET /embeddings/{entityId}| -- fetch embedding(s).
  \item \verb|POST /rsvp/step| -- advance RSVP fields and embeddings by one timestep.
\end{itemize}

%%%%%%%%%%%%%%%%%%%%%%%%%%%%%%%%%%%%%%%%%%%%%%%%%%%%%%%%%%%%%%%%%%%%%%
\section{Simulation Appendix: RSVP Dynamics on the Graph}

We sketch a discrete-time simulation to evolve RSVP fields and embeddings.

\subsection{Discrete State}

Let $G=(V,E)$ be the content graph (nodes $V$ = entities, edges $E$ = links).
At each node $i\in V$ we maintain:

\[
\Phi_i^t,\quad \mathbf{v}_i^t,\quad S_i^t,\quad \mathbf{z}_i^t.
\]

\subsection{Update Equations (Example)}

For time step $\Delta t$:

\begin{align*}
\Phi_i^{t+\Delta t}
  &= \Phi_i^t + \Delta t \left(
        D_\Phi \sum_{j\sim i} (\Phi_j^t - \Phi_i^t)
        - \lambda_\Phi \Phi_i^t
        + f_\Phi(\rho_i,\kappa_i)
     \right),\\
S_i^{t+\Delta t}
  &= S_i^t + \Delta t \left(
        D_S \sum_{j\sim i} (S_j^t - S_i^t)
        + f_S(\rho_i,\kappa_i)
     \right),\\
\mathbf{v}_i^{t+\Delta t}
  &= \mathbf{v}_i^t + \Delta t \left(
        D_v \sum_{j\sim i} (\mathbf{v}_j^t - \mathbf{v}_i^t)
        - \nabla \Phi_i^t
        - \eta \mathbf{v}_i^t
     \right),\\
\mathbf{z}_i^{t+\Delta t}
  &= \mathbf{z}_i^t
     + \Delta t \left(
          -\alpha \nabla \Phi_i^t
          + \beta \mathbf{v}_i^t
          - \gamma \nabla S_i^t
       \right),
\end{align*}

where $j\sim i$ denotes neighbors in the graph, and $\rho_i,\kappa_i$
are local density/curvature estimates.

\subsection{Pseudocode}

\begin{lstlisting}[language=Python,basicstyle=\ttfamily\small]
for t in range(T):
    # compute local graph Laplacians, densities, curvatures
    for i in V:
        lap_Phi[i] = sum(Phi[j] - Phi[i] for j in neighbors[i])
        lap_S[i]   = sum(S[j]   - S[i]   for j in neighbors[i])
        lap_v[i]   = sum(v[j]   - v[i]   for j in neighbors[i])
        rho[i]     = density_estimate(i)
        kappa[i]   = curvature_estimate(i)

    # update fields
    for i in V:
        Phi[i] += dt * (D_Phi * lap_Phi[i] - lambda_Phi * Phi[i] + f_Phi(rho[i],kappa[i]))
        S[i]   += dt * (D_S   * lap_S[i]   + f_S(rho[i],kappa[i]))
        v[i]   += dt * (D_v   * lap_v[i]   - grad_Phi[i] - eta * v[i])

    # update embeddings
    for i in V:
        z[i] += dt * (-alpha * grad_Phi[i] + beta * v[i] - gamma * grad_S[i])
\end{lstlisting}

The projection $\mathbf{x}_i = \pi(\mathbf{z}_i)$ and N-body relaxation
are performed periodically (e.g. every $K$ timesteps or after a reset).

%%%%%%%%%%%%%%%%%%%%%%%%%%%%%%%%%%%%%%%%%%%%%%%%%%%%%%%%%%%%%%%%%%%%%%
\section{Additional UML Sketches}

\subsection{Sequence: Polycompile and Reset}

The following is a textual UML-style sequence sketch (you can convert to \texttt{tikz-uml} as desired):

\begin{center}
\begin{tikzpicture}
\umlseqdiag[x=0,y=0]
  \umlactor[class=User]{user}
  \umlobject[class=GalaxyView]{galaxy}
  \umlobject[class=Polycompiler]{poly}
  \umlobject[class=RSVPFieldEngine]{engine}

  \begin{umlcall}[op=createAtom(), return=atomId]{user}{galaxy}
  \end{umlcall}
  \begin{umlcall}[op=polycompile(atomId,spec), return=variantId]{galaxy}{poly}
  \end{umlcall}
  \begin{umlcall}[op=updateFields(), return=ok]{galaxy}{engine}
  \end{umlcall}
  \begin{umlcall}[op=pressG(duration), return=resetDone]{user}{galaxy}
  \end{umlcall}
\end{tikzpicture}
\end{center}

%%%%%%%%%%%%%%%%%%%%%%%%%%%%%%%%%%%%%%%%%%%%%%%%%%%%%%%%%%%%%%%%%%%%%%
\section{Verification-Oriented Invariants and Proof Sketches}

We outline properties suitable for formal verification (e.g. Coq/Lean).

\subsection{Invariant: Link Bidirectionality}

\begin{description}
\item[Property.] For every stored TypedLink $L$ with $(\text{from}=X,\text{to}=Y)$,
the adjacency indices satisfy:
\[
Y \in \mathrm{succ}(X) \iff X \in \mathrm{pred}(Y).
\]
\end{description}

\begin{proof}[Sketch]
By construction: insertion of links is atomic and updates both
\texttt{succ} and \texttt{pred} indices. Deletion is symmetric.
Inductive reasoning over link operations proves preservation.
\end{proof}

\subsection{Invariant: Sheaf Compatibility of Galaxy Views}

\begin{description}
\item[Property.] For any two users $u,v$ with $U_u\cap U_v \neq \varnothing$:
\[
\rho_{U_u\cap U_v}(G_u) = \rho_{U_u\cap U_v}(G_v),
\]
where $G_u\in\mathscr{G}(U_u)$, $G_v\in\mathscr{G}(U_v)$ are galaxy views
derived from the same global embedding/field configuration.
\end{description}

\subsection{Safety Property: Reset Preserves Connectivity}

\begin{description}
\item[Property.] A \texttt{g}-reset does not change the content graph:
\[
\mathcal{C}' = \mathcal{C},
\]
i.e. Atoms, Spans, and Links are unchanged.
\end{description}

\begin{proof}[Sketch]
By definition of $\mathsf{Reset}$, only $\mathbf{Z},\mathbf{X},F$ are recomputed.
No content insertion/deletion occurs. Thus connectivity is preserved.
\end{proof}

\subsection{Coq/Lean-Style Theorem Template}

In a Coq-like pseudo-syntax:

\begin{verbatim}
Record State := {
  C : ContentGraph;
  Z : Embeddings;
  X : Layout;
  F : Fields;
  U : UserMeta;
}.

Inductive step : State -> State -> Prop :=
| StepGShort : forall s s',
    g_press_short s s' ->
    step s s'
| StepGLong : forall s s',
    g_press_long s s' ->
    step s s'.

Theorem reset_preserves_graph :
  forall s s',
    step s s' ->
    C s' = C s.
\end{verbatim}

A similar structure can encode invariants about autoblink pinning:

\begin{verbatim}
Definition autoblink_invariant (s s' : State) : Prop :=
  forall u, autoblink u = true ->
    dist (ship_pos s u) (ship_pos s' u) <= eps.
\end{verbatim}

%%%%%%%%%%%%%%%%%%%%%%%%%%%%%%%%%%%%%%%%%%%%%%%%%%%%%%%%%%%%%%%%%%%%%%
\section{Conclusion}

We have specified a unified architecture for Polyxan--RSVP Starspace:
a Xanadu-inspired hypermedia graph coupled to RSVP fields, projected into
a semantic starspace with user-local galaxies, ships, holographic projections,
and a global \texttt{g}-reset operator.
We provided a Lagrangian for the RSVP fields, a category-theoretic framing,
sheaf-theoretic consistency conditions for galaxy views, an operational semantics
for reset events, a database and API sketch, simulation pseudocode, UML sketches,
and verification-oriented invariants.
This document is intended as a scaffold for both theoretical refinement and
practical implementation.
\end{document}
