\documentclass[11pt]{article}

\usepackage{amsmath,amssymb,amsthm,amsfonts,mathtools}
\usepackage{bm}
\usepackage{stmaryrd}
\usepackage{hyperref}
\usepackage{geometry}
\usepackage{tikz}
\usepackage{tikz-cd}
\usepackage{mathrsfs}
\usepackage{enumitem}
\geometry{margin=1in}

\setlength{\parskip}{1em}
\setlength{\parindent}{0pt}

% ------------------------------------------------------------
% THEORETICAL MONOGRAPH
% ------------------------------------------------------------

\title{\Huge\bfseries RSVP--Polyxan:\\
A Unified Field Theory of Semantic Hyperstructures}

\author{Flyxion}
\date{2025}

\begin{document}

\maketitle

\begin{abstract}
This paper develops a unified theoretical framework coupling the
Relativistic Scalar--Vector Plenum (RSVP) field system, defined on a
semantic manifold of embeddings, with the Polyxan hypermedia substrate,
a typed, bidirectionally linked graph of content atoms and media quines
in the Xanadu tradition.
We construct an action functional for RSVP fields over semantic space,
derive Euler--Lagrange equations, introduce a geometric interpretation of
polysemantic clusters, and establish a sheaf-theoretic treatment of 
galaxy-local worldviews.
We then develop a categorical architecture for Polyxan transformations,
including a Polycompiler endofunctor and an RSVP-to-field category 
mapping.
Throughout, we present proofs, lemmas, and structural invariants that 
guarantee coherence between continuous semantic fields and discrete 
hypergraphs.
\end{abstract}

\tableofcontents

% =============================================================
\section{Introduction}

The RSVP--Polyxan framework fuses three mathematical layers:

\begin{itemize}[leftmargin=2em]
  \item A \textbf{semantic manifold} $X$ where conceptual embeddings live.
  \item A \textbf{field theory} consisting of a scalar potential $\Phi$,
    vector flow $\mathbf{v}$, and entropy field $S$.
  \item A \textbf{typed hypergraph} of content atoms and links, whose
    structure both sources and responds to RSVP fields.
\end{itemize}

The goal is to provide a unified variational, geometric,
and categorical description of semantic information flow.

% =============================================================
\section{Semantic Manifold and RSVP Fields}

Let $(X,g)$ be a smooth Riemannian manifold representing semantic space.

Define three RSVP fields:

\[
\Phi: X\to\mathbb{R},\quad
\mathbf{v}: X\to T X,\quad
S: X\to \mathbb{R}.
\]

These fields describe:

\begin{itemize}
  \item $\Phi$: semantic potential or coherence density.
  \item $\mathbf{v}$: directed agency or semantic drift.
  \item $S$: entropy, uncertainty, or morphic degeneracy.
\end{itemize}

Field derivatives are defined using the Levi-Civita connection.

% =============================================================
\section{RSVP Lagrangian Formulation}

We propose an action functional:

\[
\mathcal{A}[\Phi,\mathbf{v},S]
= \int_X \mathcal{L}\, d\mu_g
\]

where the Lagrangian density incorporates kinetic, elastic, and potential
terms:

\begin{align*}
\mathcal{L}
&= \frac{1}{2}(\partial_t\Phi)^2
 - \frac{c_\Phi^2}{2}\|\nabla\Phi\|^2 \\
&\quad + \frac{1}{2}\|\partial_t \mathbf{v}\|^2
 - \frac{c_v^2}{2}\|\nabla\mathbf{v}\|^2 \\
&\quad + \frac{1}{2}(\partial_t S)^2
 - \frac{c_S^2}{2}\|\nabla S\|^2 \\
&\quad - V(\Phi,\mathbf{v},S; \rho, \kappa).
\end{align*}

Here:

\begin{itemize}
  \item $\rho(x)$ = node density induced by Polyxan graph.
  \item $\kappa(x)$ = local curvature induced by typed links.
\end{itemize}

Variation yields Euler--Lagrange equations:

\[
\frac{\delta \mathcal{A}}{\delta \Phi}
= \partial_t^2\Phi - c_\Phi^2\Delta \Phi - \frac{\partial V}{\partial\Phi} = 0,
\]

and similarly for $\mathbf{v}, S$.

% =============================================================
\section{Coupling RSVP to the Polyxan Hypergraph}

Let $G=(V,E)$ be the Polyxan graph.
Each node $i\in V$ is assigned an embedding $\mathbf{z}_i\in X$.

Define discrete samples:

\[
\Phi_i = \Phi(\mathbf{z}_i),\quad 
\mathbf{v}_i = \mathbf{v}(\mathbf{z}_i),\quad
S_i = S(\mathbf{z}_i).
\]

Define a discrete evolution law:

\[
\frac{d\mathbf{z}_i}{dt}
= -\alpha \nabla \Phi(\mathbf{z}_i)
+ \beta\, \mathbf{v}(\mathbf{z}_i)
- \gamma \nabla S(\mathbf{z}_i).
\]

Thus embeddings evolve in RSVP gradient flows.

\subsection{Topological Curvature from Links}

Define a link-curvature invariant:

\[
\kappa_i = \sum_{j,k \sim i} f_{\text{tri}}(i,j,k)
\]

counting typed 3-cycles weighted by link types.

This curvature term shapes the potential $V$.

% =============================================================
\section{Sheaf-Theoretic Interpretation}

Define for each user $u$ an open neighborhood $U_u\subset X$.

Define a presheaf $\mathscr{G}$ of galaxy renderings:

\[
\mathscr{G}(U)=\{\text{layout functions over }U\}.
\]

Restriction maps obey:

\[
\rho_{UV}(G_U)=G_V \quad \text{for }V\subset U.
\]

\begin{theorem}
$\mathscr{G}$ is a sheaf if:
\begin{itemize}
  \item the layout map is determined by $(\Phi,\mathbf{v},S)$,
  \item embeddings are globally indexed,
  \item projections are deterministic.
\end{itemize}
\end{theorem}

\begin{proof}[Sketch]
Local layouts glued along intersections produce a unique global layout due to
determinism and continuity.  
\end{proof}

% =============================================================
\section{Category-Theoretic Architecture}

Define category $\mathbf{C}$:

\begin{itemize}
  \item Objects = Content Atoms, Spans.
  \item Morphisms = Typed Links.
\end{itemize}

Define the Polycompiler as an endofunctor:

\[
\mathsf{Poly}: \mathbf{C}\to\mathbf{C}.
\]

Define RSVP as a functor $\mathsf{RSVP}: \mathbf{C}\to\mathbf{F}$
where $\mathbf{F}$ is the category of field configurations.

\begin{theorem}
The composition $\mathsf{RSVP}\circ \mathsf{Poly}$ yields a natural
transformation describing generative semantic deformation.
\end{theorem}

% =============================================================
\section{Stability and Energy Minimization}

Define energy:

\[
E = \int_X V(\Phi,\mathbf{v},S;\rho,\kappa)\, d\mu_g.
\]

We show:

\begin{lemma}
Fixed points of RSVP flow minimize $E$ subject to embedding constraints.
\end{lemma}

\begin{proof}[Sketch]
Gradient descent of embeddings follows $-\nabla E$ by design.
\end{proof}

% =============================================================
\section{Global Reset as Field Reconfiguration}

Define reset operator:

\[
\mathcal{R}: (\Phi,\mathbf{v},S,\mathbf{z})\mapsto
(\Phi',\mathbf{v}',S',\mathbf{z}')
\]

where $\mathcal{R}$ recomputes fields and embeddings by:

\begin{itemize}
  \item re-estimating $\rho,\kappa$,
  \item relaxing the field equations,
  \item reprojecting embeddings into $X$.
\end{itemize}

Reset corresponds to a global reconfiguration of semantic geometry.

% =============================================================
\section{Conclusion}

This paper establishes RSVP--Polyxan as a unified field theory of 
semantic hyperstructures: a continuous-discrete coupling between 
fields and hypergraphs, governed by a variational principle and 
coherently assembled through sheaf and category theory.

\end{document}
